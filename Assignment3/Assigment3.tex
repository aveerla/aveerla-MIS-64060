% Options for packages loaded elsewhere
\PassOptionsToPackage{unicode}{hyperref}
\PassOptionsToPackage{hyphens}{url}
%
\documentclass[
]{article}
\usepackage{amsmath,amssymb}
\usepackage{lmodern}
\usepackage{ifxetex,ifluatex}
\ifnum 0\ifxetex 1\fi\ifluatex 1\fi=0 % if pdftex
  \usepackage[T1]{fontenc}
  \usepackage[utf8]{inputenc}
  \usepackage{textcomp} % provide euro and other symbols
\else % if luatex or xetex
  \usepackage{unicode-math}
  \defaultfontfeatures{Scale=MatchLowercase}
  \defaultfontfeatures[\rmfamily]{Ligatures=TeX,Scale=1}
\fi
% Use upquote if available, for straight quotes in verbatim environments
\IfFileExists{upquote.sty}{\usepackage{upquote}}{}
\IfFileExists{microtype.sty}{% use microtype if available
  \usepackage[]{microtype}
  \UseMicrotypeSet[protrusion]{basicmath} % disable protrusion for tt fonts
}{}
\makeatletter
\@ifundefined{KOMAClassName}{% if non-KOMA class
  \IfFileExists{parskip.sty}{%
    \usepackage{parskip}
  }{% else
    \setlength{\parindent}{0pt}
    \setlength{\parskip}{6pt plus 2pt minus 1pt}}
}{% if KOMA class
  \KOMAoptions{parskip=half}}
\makeatother
\usepackage{xcolor}
\IfFileExists{xurl.sty}{\usepackage{xurl}}{} % add URL line breaks if available
\IfFileExists{bookmark.sty}{\usepackage{bookmark}}{\usepackage{hyperref}}
\hypersetup{
  pdftitle={Assignment3},
  hidelinks,
  pdfcreator={LaTeX via pandoc}}
\urlstyle{same} % disable monospaced font for URLs
\usepackage[margin=1in]{geometry}
\usepackage{color}
\usepackage{fancyvrb}
\newcommand{\VerbBar}{|}
\newcommand{\VERB}{\Verb[commandchars=\\\{\}]}
\DefineVerbatimEnvironment{Highlighting}{Verbatim}{commandchars=\\\{\}}
% Add ',fontsize=\small' for more characters per line
\usepackage{framed}
\definecolor{shadecolor}{RGB}{248,248,248}
\newenvironment{Shaded}{\begin{snugshade}}{\end{snugshade}}
\newcommand{\AlertTok}[1]{\textcolor[rgb]{0.94,0.16,0.16}{#1}}
\newcommand{\AnnotationTok}[1]{\textcolor[rgb]{0.56,0.35,0.01}{\textbf{\textit{#1}}}}
\newcommand{\AttributeTok}[1]{\textcolor[rgb]{0.77,0.63,0.00}{#1}}
\newcommand{\BaseNTok}[1]{\textcolor[rgb]{0.00,0.00,0.81}{#1}}
\newcommand{\BuiltInTok}[1]{#1}
\newcommand{\CharTok}[1]{\textcolor[rgb]{0.31,0.60,0.02}{#1}}
\newcommand{\CommentTok}[1]{\textcolor[rgb]{0.56,0.35,0.01}{\textit{#1}}}
\newcommand{\CommentVarTok}[1]{\textcolor[rgb]{0.56,0.35,0.01}{\textbf{\textit{#1}}}}
\newcommand{\ConstantTok}[1]{\textcolor[rgb]{0.00,0.00,0.00}{#1}}
\newcommand{\ControlFlowTok}[1]{\textcolor[rgb]{0.13,0.29,0.53}{\textbf{#1}}}
\newcommand{\DataTypeTok}[1]{\textcolor[rgb]{0.13,0.29,0.53}{#1}}
\newcommand{\DecValTok}[1]{\textcolor[rgb]{0.00,0.00,0.81}{#1}}
\newcommand{\DocumentationTok}[1]{\textcolor[rgb]{0.56,0.35,0.01}{\textbf{\textit{#1}}}}
\newcommand{\ErrorTok}[1]{\textcolor[rgb]{0.64,0.00,0.00}{\textbf{#1}}}
\newcommand{\ExtensionTok}[1]{#1}
\newcommand{\FloatTok}[1]{\textcolor[rgb]{0.00,0.00,0.81}{#1}}
\newcommand{\FunctionTok}[1]{\textcolor[rgb]{0.00,0.00,0.00}{#1}}
\newcommand{\ImportTok}[1]{#1}
\newcommand{\InformationTok}[1]{\textcolor[rgb]{0.56,0.35,0.01}{\textbf{\textit{#1}}}}
\newcommand{\KeywordTok}[1]{\textcolor[rgb]{0.13,0.29,0.53}{\textbf{#1}}}
\newcommand{\NormalTok}[1]{#1}
\newcommand{\OperatorTok}[1]{\textcolor[rgb]{0.81,0.36,0.00}{\textbf{#1}}}
\newcommand{\OtherTok}[1]{\textcolor[rgb]{0.56,0.35,0.01}{#1}}
\newcommand{\PreprocessorTok}[1]{\textcolor[rgb]{0.56,0.35,0.01}{\textit{#1}}}
\newcommand{\RegionMarkerTok}[1]{#1}
\newcommand{\SpecialCharTok}[1]{\textcolor[rgb]{0.00,0.00,0.00}{#1}}
\newcommand{\SpecialStringTok}[1]{\textcolor[rgb]{0.31,0.60,0.02}{#1}}
\newcommand{\StringTok}[1]{\textcolor[rgb]{0.31,0.60,0.02}{#1}}
\newcommand{\VariableTok}[1]{\textcolor[rgb]{0.00,0.00,0.00}{#1}}
\newcommand{\VerbatimStringTok}[1]{\textcolor[rgb]{0.31,0.60,0.02}{#1}}
\newcommand{\WarningTok}[1]{\textcolor[rgb]{0.56,0.35,0.01}{\textbf{\textit{#1}}}}
\usepackage{graphicx}
\makeatletter
\def\maxwidth{\ifdim\Gin@nat@width>\linewidth\linewidth\else\Gin@nat@width\fi}
\def\maxheight{\ifdim\Gin@nat@height>\textheight\textheight\else\Gin@nat@height\fi}
\makeatother
% Scale images if necessary, so that they will not overflow the page
% margins by default, and it is still possible to overwrite the defaults
% using explicit options in \includegraphics[width, height, ...]{}
\setkeys{Gin}{width=\maxwidth,height=\maxheight,keepaspectratio}
% Set default figure placement to htbp
\makeatletter
\def\fps@figure{htbp}
\makeatother
\setlength{\emergencystretch}{3em} % prevent overfull lines
\providecommand{\tightlist}{%
  \setlength{\itemsep}{0pt}\setlength{\parskip}{0pt}}
\setcounter{secnumdepth}{-\maxdimen} % remove section numbering
\ifluatex
  \usepackage{selnolig}  % disable illegal ligatures
\fi

\title{Assignment3}
\author{}
\date{\vspace{-2.5em}}

\begin{document}
\maketitle

\hypertarget{r-markdown}{%
\subsection{R Markdown}\label{r-markdown}}

This is an R Markdown document. Markdown is a simple formatting syntax
for authoring HTML, PDF, and MS Word documents. For more details on
using R Markdown see \url{http://rmarkdown.rstudio.com}.

When you click the \textbf{Knit} button a document will be generated
that includes both content as well as the output of any embedded R code
chunks within the document. You can embed an R code chunk like this:

\begin{Shaded}
\begin{Highlighting}[]
\CommentTok{\#Installing libraries }
\FunctionTok{library}\NormalTok{(reshape2)}
\FunctionTok{library}\NormalTok{(gmodels)}
\FunctionTok{library}\NormalTok{(caret)}
\end{Highlighting}
\end{Shaded}

\begin{verbatim}
## Loading required package: lattice
\end{verbatim}

\begin{verbatim}
## Loading required package: ggplot2
\end{verbatim}

\begin{Shaded}
\begin{Highlighting}[]
\FunctionTok{library}\NormalTok{(ISLR)}
\FunctionTok{library}\NormalTok{(e1071)}
\end{Highlighting}
\end{Shaded}

\begin{Shaded}
\begin{Highlighting}[]
\CommentTok{\#Read universalbank CSV file}
\NormalTok{UnivBank }\OtherTok{\textless{}{-}}\FunctionTok{read.csv}\NormalTok{(}\StringTok{"UniversalBank.CSV"}\NormalTok{)}
\end{Highlighting}
\end{Shaded}

\begin{Shaded}
\begin{Highlighting}[]
\CommentTok{\#conerting variables}
\NormalTok{UnivBank}\SpecialCharTok{$}\NormalTok{Personal.Loan}\OtherTok{\textless{}{-}}\FunctionTok{factor}\NormalTok{(UnivBank}\SpecialCharTok{$}\NormalTok{Personal.Loan)}
\NormalTok{UnivBank}\SpecialCharTok{$}\NormalTok{Online}\OtherTok{\textless{}{-}}\FunctionTok{factor}\NormalTok{(UnivBank}\SpecialCharTok{$}\NormalTok{Online)}
\NormalTok{UnivBank}\SpecialCharTok{$}\NormalTok{CreditCard}\OtherTok{\textless{}{-}}\FunctionTok{factor}\NormalTok{(UnivBank}\SpecialCharTok{$}\NormalTok{CreditCard)}
\end{Highlighting}
\end{Shaded}

\begin{Shaded}
\begin{Highlighting}[]
\FunctionTok{set.seed}\NormalTok{(}\DecValTok{10}\NormalTok{)}
\CommentTok{\#Spliting data into training 60\% and validation 40\%}
\NormalTok{t.index }\OtherTok{\textless{}{-}}\FunctionTok{sample}\NormalTok{(}\FunctionTok{row.names}\NormalTok{(UnivBank), }\FloatTok{0.6}\SpecialCharTok{*}\FunctionTok{dim}\NormalTok{(UnivBank)[}\DecValTok{1}\NormalTok{])}
\NormalTok{validt.index }\OtherTok{\textless{}{-}} \FunctionTok{setdiff}\NormalTok{(}\FunctionTok{row.names}\NormalTok{(UnivBank), t.index)}
\NormalTok{t.df }\OtherTok{\textless{}{-}}\NormalTok{ UnivBank[t.index, ]}
\NormalTok{validt.df }\OtherTok{\textless{}{-}}\NormalTok{ UnivBank[validt.index, ]}
\NormalTok{train }\OtherTok{\textless{}{-}}\NormalTok{ UnivBank[t.index,]}
\NormalTok{validtest }\OtherTok{\textless{}{-}}\NormalTok{ UnivBank[t.index, ]}
\end{Highlighting}
\end{Shaded}

\#A Create a pivot table for the training data with Online as a column
variable, CC as a row variable, and Loan as a secondary row variable.
The values inside the table should convey the count. In R use functions
melt() and cast(), or function table(). In Python, use panda dataframe
methods melt() and pivot().

\begin{Shaded}
\begin{Highlighting}[]
\NormalTok{melt.bank }\OtherTok{\textless{}{-}} \FunctionTok{melt}\NormalTok{(train, }\AttributeTok{id=}\FunctionTok{c}\NormalTok{(}\StringTok{"CreditCard"}\NormalTok{, }\StringTok{"Personal.Loan"}\NormalTok{),}\AttributeTok{variable=}\StringTok{"Online"}\NormalTok{)}
\end{Highlighting}
\end{Shaded}

\begin{verbatim}
## Warning: attributes are not identical across measure variables; they will be
## dropped
\end{verbatim}

\begin{Shaded}
\begin{Highlighting}[]
\NormalTok{cast.bank }\OtherTok{\textless{}{-}} \FunctionTok{dcast}\NormalTok{(melt.bank,CreditCard}\SpecialCharTok{+}\NormalTok{Personal.Loan}\SpecialCharTok{\textasciitilde{}}\NormalTok{Online)}
\end{Highlighting}
\end{Shaded}

\begin{verbatim}
## Aggregation function missing: defaulting to length
\end{verbatim}

\begin{Shaded}
\begin{Highlighting}[]
\NormalTok{cast.bank[,}\FunctionTok{c}\NormalTok{(}\DecValTok{1}\SpecialCharTok{:}\DecValTok{2}\NormalTok{,}\DecValTok{14}\NormalTok{)]}
\end{Highlighting}
\end{Shaded}

\begin{verbatim}
##   CreditCard Personal.Loan Online
## 1          0             0   1923
## 2          0             1    202
## 3          1             0    782
## 4          1             1     93
\end{verbatim}

\#B Consider the task of classifying a customer who owns a bank credit
card and is actively using online banking services. Looking at the pivot
table, what is the probability that this customer will accept the loan
offer? {[}This is the probability of loan acceptance (Loan = 1)
conditional on having a bank credit card (CC = 1) and being an active
user of online banking services (Online = 1){]}.

\begin{Shaded}
\begin{Highlighting}[]
\NormalTok{x}\OtherTok{=} \FunctionTok{table}\NormalTok{(train[,}\FunctionTok{c}\NormalTok{(}\DecValTok{10}\NormalTok{,}\DecValTok{13}\NormalTok{,}\DecValTok{14}\NormalTok{)])}
\NormalTok{y}\OtherTok{\textless{}{-}}\FunctionTok{as.data.frame}\NormalTok{(x)}
\NormalTok{y}
\end{Highlighting}
\end{Shaded}

\begin{verbatim}
##   Personal.Loan Online CreditCard Freq
## 1             0      0          0  772
## 2             1      0          0   79
## 3             0      1          0 1151
## 4             1      1          0  123
## 5             0      0          1  300
## 6             1      0          1   40
## 7             0      1          1  482
## 8             1      1          1   53
\end{verbatim}

\#C Create two separate pivot tables for the training data. One will
have Loan (rows) as a function of Online (columns) and the other will
have Loan (rows) as a function of CC. \#Creating pivot table for Loan
(rows) as a function of Online (columns)

\begin{Shaded}
\begin{Highlighting}[]
\FunctionTok{table}\NormalTok{(train[,}\FunctionTok{c}\NormalTok{(}\DecValTok{10}\NormalTok{,}\DecValTok{13}\NormalTok{)])}
\end{Highlighting}
\end{Shaded}

\begin{verbatim}
##              Online
## Personal.Loan    0    1
##             0 1072 1633
##             1  119  176
\end{verbatim}

\#Creating pivot table for Loan (rows) as a function of CC

\begin{Shaded}
\begin{Highlighting}[]
\FunctionTok{table}\NormalTok{(train[,}\FunctionTok{c}\NormalTok{(}\DecValTok{10}\NormalTok{,}\DecValTok{14}\NormalTok{)])}
\end{Highlighting}
\end{Shaded}

\begin{verbatim}
##              CreditCard
## Personal.Loan    0    1
##             0 1923  782
##             1  202   93
\end{verbatim}

\#D Compute the following quantities {[}P(A \textbar{} B) means ``the
probability ofA given B''{]}: i. P(CC = 1 \textbar{} Loan = 1) (the
proportion of credit card holders among the loan acceptors) ii. P(Online
= 1 \textbar{} Loan = 1) iii. P(Loan = 1) (the proportion of loan
acceptors) iv. P(CC = 1 \textbar{} Loan = 0) v. P(Online = 1 \textbar{}
Loan = 0) vi. P(Loan = 0)

\begin{Shaded}
\begin{Highlighting}[]
\CommentTok{\#i P(CC = 1 | Loan = 1)}
\NormalTok{P1 }\OtherTok{\textless{}{-}}\FunctionTok{table}\NormalTok{(train[,}\FunctionTok{c}\NormalTok{(}\DecValTok{14}\NormalTok{,}\DecValTok{10}\NormalTok{)])}
\NormalTok{S1}\OtherTok{\textless{}{-}}\NormalTok{ P1[}\DecValTok{2}\NormalTok{,}\DecValTok{2}\NormalTok{]}\SpecialCharTok{/}\NormalTok{(P1[}\DecValTok{2}\NormalTok{,}\DecValTok{2}\NormalTok{]}\SpecialCharTok{+}\NormalTok{P1[}\DecValTok{1}\NormalTok{,}\DecValTok{2}\NormalTok{])}
\NormalTok{S1}
\end{Highlighting}
\end{Shaded}

\begin{verbatim}
## [1] 0.3152542
\end{verbatim}

\#ii P(Online = 1 \textbar{} Loan = 1)

\begin{Shaded}
\begin{Highlighting}[]
\NormalTok{P2 }\OtherTok{\textless{}{-}} \FunctionTok{table}\NormalTok{(train[, }\FunctionTok{c}\NormalTok{(}\DecValTok{13}\NormalTok{,}\DecValTok{10}\NormalTok{)])}
\NormalTok{S2 }\OtherTok{\textless{}{-}}\NormalTok{ P2[}\DecValTok{2}\NormalTok{,}\DecValTok{2}\NormalTok{]}\SpecialCharTok{/}\NormalTok{(P2[}\DecValTok{2}\NormalTok{,}\DecValTok{2}\NormalTok{]}\SpecialCharTok{+}\NormalTok{P2[}\DecValTok{1}\NormalTok{,}\DecValTok{2}\NormalTok{])}
\NormalTok{S2}
\end{Highlighting}
\end{Shaded}

\begin{verbatim}
## [1] 0.5966102
\end{verbatim}

\#iii P(Loan = 1)

\begin{Shaded}
\begin{Highlighting}[]
\NormalTok{P3}\OtherTok{\textless{}{-}}\FunctionTok{table}\NormalTok{(train[,}\DecValTok{10}\NormalTok{])}
\NormalTok{S3}\OtherTok{\textless{}{-}}\NormalTok{P3[}\DecValTok{2}\NormalTok{]}\SpecialCharTok{/}\NormalTok{(P3[}\DecValTok{2}\NormalTok{]}\SpecialCharTok{+}\NormalTok{P3[}\DecValTok{1}\NormalTok{])}
\NormalTok{S3}
\end{Highlighting}
\end{Shaded}

\begin{verbatim}
##          1 
## 0.09833333
\end{verbatim}

\#iv P(CC = 1 \textbar{} Loan = 0)

\begin{Shaded}
\begin{Highlighting}[]
\NormalTok{P4}\OtherTok{\textless{}{-}}\FunctionTok{table}\NormalTok{(train[,}\FunctionTok{c}\NormalTok{(}\DecValTok{14}\NormalTok{,}\DecValTok{10}\NormalTok{)])}
\NormalTok{S4}\OtherTok{\textless{}{-}}\NormalTok{P4[}\DecValTok{2}\NormalTok{,}\DecValTok{1}\NormalTok{]}\SpecialCharTok{/}\NormalTok{(P4[}\DecValTok{2}\NormalTok{,}\DecValTok{1}\NormalTok{]}\SpecialCharTok{+}\NormalTok{P4[}\DecValTok{1}\NormalTok{,}\DecValTok{1}\NormalTok{])}
\NormalTok{S4}
\end{Highlighting}
\end{Shaded}

\begin{verbatim}
## [1] 0.2890943
\end{verbatim}

\#v P(Online = 1 \textbar{} Loan = 0)

\begin{Shaded}
\begin{Highlighting}[]
\NormalTok{P5}\OtherTok{\textless{}{-}}\FunctionTok{table}\NormalTok{(train[,}\FunctionTok{c}\NormalTok{(}\DecValTok{13}\NormalTok{,}\DecValTok{10}\NormalTok{)])}
\NormalTok{S5}\OtherTok{\textless{}{-}}\NormalTok{P5[}\DecValTok{2}\NormalTok{,}\DecValTok{1}\NormalTok{]}\SpecialCharTok{/}\NormalTok{(P5[}\DecValTok{2}\NormalTok{,}\DecValTok{1}\NormalTok{]}\SpecialCharTok{+}\NormalTok{P5[}\DecValTok{1}\NormalTok{,}\DecValTok{1}\NormalTok{])}
\NormalTok{S5}
\end{Highlighting}
\end{Shaded}

\begin{verbatim}
## [1] 0.6036969
\end{verbatim}

\#vi P(Loan = 0)

\begin{Shaded}
\begin{Highlighting}[]
\NormalTok{P6}\OtherTok{\textless{}{-}}\FunctionTok{table}\NormalTok{(train[,}\DecValTok{10}\NormalTok{])}
\NormalTok{S6}\OtherTok{\textless{}{-}}\NormalTok{P6[}\DecValTok{1}\NormalTok{]}\SpecialCharTok{/}\NormalTok{(P6[}\DecValTok{1}\NormalTok{]}\SpecialCharTok{+}\NormalTok{P6[}\DecValTok{2}\NormalTok{])}
\NormalTok{S6}
\end{Highlighting}
\end{Shaded}

\begin{verbatim}
##         0 
## 0.9016667
\end{verbatim}

\#E Use the quantities computed above to compute the naive Bayes
probability P(Loan = 1 \textbar{} CC= 1, Online = 1).
\#NaiveBayesProbability=
(S1\emph{S2}S3)/{[}(S1\emph{S2}S3)+(S4\emph{S5}S6){]}
\#0.01849491/(0.01849491+0.15736368)=0.1051692

\#F Compare this value with the one obtained from the pivot table in
(B). Which is a more accurate estimate?

\#The value we got from pivot table is 0.092831 and the naive bayes is
0.1051692 and are almost similar. Pivot table value is more accurate.

\#G Which of the entries in this table are needed for computing P(Loan =
1 \textbar{} CC = 1, Online = 1)? Run naive Bayes on the data. Examine
the model output on training data, and find the entry that corresponds
to P(Loan = 1 \textbar{} CC = 1, Online = 1). Compare this to the number
you obtained in (E).

\#Naive Bayes on training data

\begin{Shaded}
\begin{Highlighting}[]
\FunctionTok{table}\NormalTok{(train[,}\FunctionTok{c}\NormalTok{(}\DecValTok{10}\NormalTok{,}\DecValTok{13}\SpecialCharTok{:}\DecValTok{14}\NormalTok{)])}
\end{Highlighting}
\end{Shaded}

\begin{verbatim}
## , , CreditCard = 0
## 
##              Online
## Personal.Loan    0    1
##             0  772 1151
##             1   79  123
## 
## , , CreditCard = 1
## 
##              Online
## Personal.Loan    0    1
##             0  300  482
##             1   40   53
\end{verbatim}

\begin{Shaded}
\begin{Highlighting}[]
\NormalTok{train\_Naive}\OtherTok{\textless{}{-}}\NormalTok{train[,}\FunctionTok{c}\NormalTok{(}\DecValTok{10}\NormalTok{,}\DecValTok{13}\SpecialCharTok{:}\DecValTok{14}\NormalTok{)]}
\NormalTok{UnivBank\_NB}\OtherTok{\textless{}{-}}\FunctionTok{naiveBayes}\NormalTok{(Personal.Loan}\SpecialCharTok{\textasciitilde{}}\NormalTok{.,}\AttributeTok{data =}\NormalTok{ train\_Naive)}
\NormalTok{UnivBank\_NB}
\end{Highlighting}
\end{Shaded}

\begin{verbatim}
## 
## Naive Bayes Classifier for Discrete Predictors
## 
## Call:
## naiveBayes.default(x = X, y = Y, laplace = laplace)
## 
## A-priori probabilities:
## Y
##          0          1 
## 0.90166667 0.09833333 
## 
## Conditional probabilities:
##    Online
## Y           0         1
##   0 0.3963031 0.6036969
##   1 0.4033898 0.5966102
## 
##    CreditCard
## Y           0         1
##   0 0.7109057 0.2890943
##   1 0.6847458 0.3152542
\end{verbatim}

\hypertarget{after-running-naive-bayes-on-data-value-obtained-is-0.1051692-where-as-value-from-e-is-0.1051692-which-is-almost-similar.}{%
\section{After running Naive bayes on data Value obtained is 0.1051692
where as value from E is 0.1051692 which is almost
similar.}\label{after-running-naive-bayes-on-data-value-obtained-is-0.1051692-where-as-value-from-e-is-0.1051692-which-is-almost-similar.}}

\hypertarget{including-plots}{%
\subsection{Including Plots}\label{including-plots}}

You can also embed plots, for example:

\includegraphics{Assigment3_files/figure-latex/pressure-1.pdf}

Note that the \texttt{echo\ =\ FALSE} parameter was added to the code
chunk to prevent printing of the R code that generated the plot.

\end{document}
